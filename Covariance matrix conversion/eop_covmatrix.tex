\documentclass[letter]{article}
\usepackage[utf8]{inputenc}
\usepackage{enumitem}
%\usepackage{fullpage}
\usepackage[margin=1.0in]{geometry}
%\usepackage{fancyhdr}
\setlength{\headheight}{5pt}
%\setlength{\textheight}{650pt}
%\pagestyle{fancy}
\linespread{1.0}
\setlength{\parindent}{0pt}
\usepackage{amsmath,amsfonts,amssymb,amsthm,bbm,graphicx,enumerate,times} 
\usepackage{hyperref}
\usepackage{mathtools}
\usepackage[usenames,dvipsnames]{color}
\usepackage{dsfont}

%%%================================================
%%%==============  new commands  =======================
%%%================================================
%%% colors
\newcommand{\cred}{\color{red}}
\newcommand{\cblue}{\color{blue}}
\newcommand{\cgreen}{\color{cyan}}
\newcommand{\wi}{\textcolor{green}}
\newcommand{\old}{\textcolor{blue}}

%%% mathbb
\usepackage{dsfont}
\newcommand{\id}{\mathds{1}}

\newcommand{\RR}{\mathbb{R}}
\newcommand{\CC}{\mathbb{C}}
\newcommand{\ZZ}{\mathbb{Z}}
\newcommand{\EE}{\mathbb{E}}
\renewcommand{\O}{\mathbb{O}}%orthogonal group
\newcommand{\SO}{\mathbb{SO}}

%%% mathrm
\newcommand{\tr}{\mathrm{tr}}
\newcommand{\e}{\ensuremath\mathrm{e}}
 \renewcommand{\i}{\,\ensuremath\mathrm{i}}
\newcommand{\X}{X}
\newcommand{\Y}{Y}
\newcommand{\Z}{Z}

%%% mathcal
\newcommand{\J}{\mathcal{J}}
\newcommand{\li}{\mathcal{L}}
\newcommand{\ki}{\mathcal{K}}
\newcommand{\W}{\mathcal{W}}
\newcommand{\F}{F_\W}
\renewcommand{\S}{\mathcal{S}_{n}}

% new thm environments
\newtheorem{theorem}{Theorem}
\newtheorem{lemma}[theorem]{Lemma}
\newtheorem{proposition}[theorem]{Proposition}
\newtheorem{definition}[theorem]{Definition}
\newtheorem{conjecture}[theorem]{Conjecture}
% \newtheorem{example}[theorem]{Example}
% \newtheorem{implication}[theorem]{Implication}
% \newtheorem{assumption}[theorem]{Assumption}
% \newtheorem{corollary}[theorem]{Corollary}
% \newtheorem{observation}[theorem]{Observation}

% math signs with small spacing
\newcommand{\spl}{\,{+}\,}
\newcommand{\smin}{\,{-}\,}
\newcommand{\seq}{\,{=}\,}
\newcommand{\sneq}{\,{\neq}\,}
\newcommand{\sle}{\,{<}\,}
\newcommand{\sleq}{\,{\leq}\,}
\newcommand{\sgr}{\,{>}\,}
\newcommand{\sgeq}{\,{\geq}\,}
\newcommand{\stimes}{\,{\times}\,}
\newcommand{\sto}{\,{\to}\,}
\newcommand{\sapprox}{\,{\approx}\,}
\newcommand{\selem}{\,{\in}\,}
\newcommand{\sdef}{\,{\coloneqq}\,}


%%% ============= old =================
%%% ------ rm ---------
% \newcommand{\e}{\ensuremath\mathrm{e}}
% \renewcommand{\i}{\ensuremath\mathrm{i}}
% \newcommand{\id}{\ensuremath\mathrm{id}}
% \DeclareMathOperator{\landauO}{\mathrm{O}}
% \DeclareMathOperator{\landauo}{\mathrm{o}}
\DeclareMathOperator{\Tr}{Tr}
% \renewcommand{\Re}{\operatorname{Re}}
% \renewcommand{\Im}{\operatorname{Im}}
% \DeclareMathOperator{\ran}{ran}
\DeclareMathOperator{\rank}{rank}
% \DeclareMathOperator{\vecmap}{vec}
% \DeclareMathOperator{\Pos}{Pos}
% \DeclareMathOperator*{\argmin}{arg\,min}
% \DeclareMathOperator{\supp}{supp}
% \DeclareMathOperator{\diag}{diag}
% \DeclareMathOperator{\spec}{spec}
% \DeclareMathOperator{\dist}{dist}
% \newcommand{\fro}{\mathrm{F}}
% \renewcommand{\ker}{\operatorname{ker}}
% \DeclareMathOperator{\cov}{\operatorname{cov}}
% \DeclareMathOperator{\eig}{\operatorname{eig}}
% \newcommand{\hc}{\mathrm{h.c.}}


%%% ------ other ----------
% \newcommand{\kw}[1]{\frac{1}{#1}}
% \newcommand{\tkw}[1]{\tfrac{1}{#1}}
\newcommand{\argdot}{{\,\cdot\,}}
% \renewcommand{\vec}[1]{\mathbf{#1}}
% \providecommand{\DC}{\operatorname{\mathscr{D}}} %descent cone
% \newcommand{\ad}{^\dagger}%symbol for conjugate transpose
% \renewcommand{\L}{\operatorname{\mathrm{L}}}%linear operators
% \newcommand{\LL}{\operatorname{\mathbb{L}}}%linear maps on operators
% % \newcommand{\M}{\operatorname{\mathbb{L}}}
% \newcommand{\DM}{\operatorname{\mc{D}}}
% \DeclareMathOperator{\Herm}{Herm}
% \newcommand{\restr}{\upharpoonright}
%\renewcommand{\ol}[1]{\overline{#1}}
% \newcommand{\inner}[1]{\mathring{#1}}
% \DeclareMathOperator{\dunion}{\uplus}%Disjoint union
% \DeclareMathOperator{\dUnion}{\biguplus}%Disjoint union
% \renewcommand{\complement}{^{c}}
% \DeclareMathOperator{\colonequiv}{:\!\Leftrightarrow}


%%% ------ norms, inner product ----------
\newcommand{\norm}[1]{\left\Vert #1 \right\Vert} %norm with variable height
\newcommand{\normn}[1]{\lVert #1 \rVert} %norm with normalsize height
\newcommand{\normb}[1]{\bigl\Vert #1 \bigr\Vert} %norm with big height
\newcommand{\normB}[1]{\Bigl\Vert #1 \Bigr\Vert} %norm with Big height
\newcommand{\mnorm}[1]{\norm{#1}_{\max{}}} % max norm
\newcommand{\mnormn}[1]{\normn{#1}_{\max{}}}
\newcommand{\mnormb}[1]{\normb{#1}_{\max{}}}
\newcommand{\mnormB}[1]{\normB{#1}_{\max{}}}
\newcommand{\snorm}[1]{\norm{#1}_\infty} %spectral norm  =  (2->2)-norm
\newcommand{\snormn}[1]{\normn{#1}_\infty}
\newcommand{\snormb}[1]{\normb{#1}_\infty}
\newcommand{\snormB}[1]{\normB{#1}_\infty}
% \newcommand{\tnorm}[1]{\norm{#1}_{1}} %1 norm
% \newcommand{\tnormn}[1]{\normn{#1}_{1}}
% \newcommand{\tnormb}[1]{\normb{#1}_{1}}
% \newcommand{\fnorm}[1]{\norm{#1}_\fro} %2 norm
% \newcommand{\fnormn}[1]{\normn{#1}_\fro}
% \newcommand{\fnormb}[1]{\normb{#1}_\fro}
%\newcommand{\nnorm}[1]{\norm{#1}_\ast} %nuclear norm
%\newcommand{\nnormn}[1]{\normn{#1}_\ast}
%\newcommand{\nnormb}[1]{\normb{#1}_\ast}

%%% ---- Kets -----
\newcommand{\ket}[1]{\left.\left|{#1}\right.\right\rangle}
\newcommand{\sket}[1]{\left.|{#1}\right.\rangle}
\newcommand{\kett}[1]{\vert{#1}\rangle}
\newcommand{\ketn}[1]{| #1 \rangle}
\newcommand{\ketb}[1]{\bigl| #1 \bigr\rangle}
\newcommand{\bra}[1]{\left.\left\langle{#1}\right.\right|}
\newcommand{\bran}[1]{\langle #1 |}
\newcommand{\braket}[2]{\left\langle #1 \middle| #2 \right\rangle}
\newcommand{\braketn}[2]{\langle #1 \middle| #2 \rangle}
\newcommand{\ketbra}[2]{\ket{#1} \!\! \bra{#2}}
\newcommand{\ketibra}[3]{\ket{#1}_{#2} \!\! \bra{#3}}
\newcommand{\ketbran}[2]{\ketn{#1} \! \bran{#2}}
\newcommand{\sandwich}[3]
  {\left\langle  #1 \right| #2 \left| #3 \right\rangle}
\newcommand{\sandwichb}[3]
  {\bigl\langle  #1 \bigr| #2 \bigl| #3 \bigr\rangle}
%vacuum
\newcommand\vac{{\ketbra{\emptyset}{\emptyset}}}
\newcommand\vacbra{{\bra{\emptyset}}}
\newcommand\vacket{{\ket{\emptyset}}}


%%% ------ specific to this project ----------
\newcommand{\ma}[1]{{m^{(1)}_{#1}}}
\newcommand{\mb}[1]{{m^{(2)}_{#1}}}
\newcommand{\ms}[1]{{m^{(\sigma)}_{#1}}}
\newcommand{\gs}[2]{{i\ms{#1}\ms{#2}}}
\DeclareMathAlphabet{\mathpzcc}{OT1}{pzc}{m}{it}
\DeclareMathAlphabet{\mathpzc}{T1}{pzc}{m}{it}{\huge}
\newcommand{\msp}{\phantom{-}}
\newcommand{\m}{\operatorname{\gamma}}
\newcommand{\mt}{\operatorname{\tilde{\gamma}}}
\newcommand{\fe}{\operatorname{f}^{\phantom{\dagger}}}
\newcommand{\ft}{\tilde{\operatorname{f}}^{\phantom{\dagger}}}
\newcommand{\fd}{\operatorname{f}^\dagger}
\newcommand{\ftd}{\tilde{\operatorname{f}}^\dagger}
  \newcommand{\M} {\mathpzcc{m}}
  \newcommand{\s}{ {|S|}}
  \newcommand{\supp}{\mathrm{supp}}
  \def\tildevac{\kett{\widetilde \emptyset}}
\newcommand\Par[1][]{%
  \ifstrempty{#1}{%
    \mathcal P_\text{tot}
  }{
    \mathcal P_{#1}
  }
}
\newcommand{\Ai}[1]{T^{(#1)}}
\newcommand{\pentagon}{\tikz{
\node[draw,regular polygon, regular polygon sides=5, inner sep=0pt,minimum size=8pt]at (0,0){};
}}

\newcommand{\pentapic}[1]{{\tikz{
\node[name=P,draw,regular polygon, regular polygon sides=5, inner sep=0pt,minimum size=8pt]at (0,0){};
\node[inner sep=0pt](P.center){\tiny $#1$}
}}}
\newcommand{\penta}[1]{\ket{\pentapic{#1}}}
\def\pent{\ket{\pentagon}}
\def\twopent{\pentapic 1,\pentapic 2}
\def\twopentc{\pentapic 1 \times\pentapic 2}
\def\JW{JW}
\newcommand{\p}[1]{p^{(#1)}}
\newcommand{\contr}[2]{C^{(#1\rightarrow #2)}_{i,j}}
\def\psit{\ket{\Psi}}

\newcommand\Pf[1]{\mathrm{Pf}\left(#1\right)}
\def\fc{{\operatorname{f}^{(c)}}}
\def\fct{{{\tilde \operatorname{f}}^{(c)}}}
\def\fa{{\operatorname{f}^{(a)}}}
\def\f{{\operatorname{f}^{(full)}}}
\def\nUG{{ :U_G:}}
\def\for{\quad\text{for}\quad}
\def\and{\quad\text{and}\quad}
\def\with{\quad\text{with}\quad}
\def\where{\quad\text{where}\quad}
\def\xor{\ \texttt{XOR}\ }
\newcommand\nU[1]{ {: U_{#1} :}}
\newcommand\BC[1]{ {\{0,1\}^{\times #1}}}
\def\BCL{ \{0,1\}^{\times L}}
\def\BCR{ \{0,1\}^{\times r}}
\def\d{\text{d}}



\newcommand{\intd}[1]{\int\mathrm{d}#1\,}

\begin{document}

\section{Covariance matrices and scalar fields}
\subsection{Real case}

In the study of entanglement of purification (EoP), we considered mixed states with density matrices of the form
\begin{equation}
\label{EQ_RHO1}
\rho(\phi, \phi^\prime) \propto \exp\left[ 
-\frac{1}{2}\;
\begin{pmatrix}
\phi &
\phi^\prime
\end{pmatrix}
\begin{pmatrix}
P - \frac{1}{2}Q R^{-1} Q^\mathrm{T} & - \frac{1}{2}Q R^{-1} Q^\mathrm{T}\\
- \frac{1}{2}Q R^{-1} Q^\mathrm{T} & P - \frac{1}{2}Q R^{-1} Q^\mathrm{T}
\end{pmatrix}
\begin{pmatrix}
\phi \\
\phi^\prime
\end{pmatrix}
 \right]\ .
\end{equation}
The bosonic field $\phi$ is discretized on a lattice of $n$ sites in the form $\phi(x^k) \equiv \phi^k$, with conjugate momentum $\pi^k$ obeying $[\phi^j,\pi^k] = \i\delta^{j k}$. The real $n\times n$ matrices $P,Q,R$ depend on the scalar field ansatz and the position of the sites that are traced out.
In general, we can write density matrices of the form \eqref{EQ_RHO1} as
\begin{equation}
\label{EQ_RHO2}
\rho(\phi, \phi^\prime) = \sqrt{\det\frac{A+B}{\pi}} e^{-\frac{1}{2} (\phi^\mathrm{T} A \phi + {\phi^\prime}^\mathrm{T} A \phi^\prime + \phi^\mathrm{T} B \phi^\prime  + {\phi^\prime}^\mathrm{T} B \phi) }\ ,
\end{equation}
with real symmetric matrices $A,B$. To check that this density matrix is properly normalized, we compute
\begin{align}
Z = \tr{\rho} &=  \sqrt{\det\frac{A+B}{\pi}} \intd\phi \intd\phi^\prime  \delta^{(n)}(\phi - \phi^\prime)\, e^{ -\frac{1}{2} (\phi^\mathrm{T} A \phi + {\phi^\prime}^\mathrm{T} A \phi^\prime + \phi^\mathrm{T} B \phi^\prime + {\phi^\prime}^\mathrm{T} B \phi ) } \nonumber\\
&= \sqrt{\det\frac{A+B}{\pi}} \intd\phi e^{ -\phi^\mathrm{T} (A+B) \phi} 
= 1\ .
\end{align}
Here we used the notation $\intd\phi = \prod_k \intd{\phi_k}$ and $\delta^{(n)}(\phi - \phi^\prime) = \prod_k \delta(\phi_k - \phi_k^\prime)$. We are now interested in computing the \emph{covariance matrix}
\begin{equation}
G^{a b} = \langle \{ \xi^a, \xi^b \} \rangle =
\begin{pmatrix}
\langle \{ \phi^\alpha, \phi^\beta \} \rangle & \langle \{ \phi^\alpha, \pi^\beta \} \rangle \\
\langle \{ \pi^\alpha, \phi^\beta \} \rangle & \langle \{ \pi^\alpha, \pi^\beta \} \rangle
\end{pmatrix}
\equiv
\begin{pmatrix}
\Gamma^{\alpha\beta}_{\phi\phi} & \Gamma^{\alpha\beta}_{\phi\pi} \\
\Gamma^{\beta\alpha}_{\phi\pi} & \Gamma^{\alpha\beta}_{\pi\pi}
\end{pmatrix}\ .
\end{equation}
We introduced the quadrature coordinates $\xi^a = (\phi^1, \phi^2,\dots,\phi^n,\pi^1,\pi^2,\dots,\pi^n)$. To find the $\phi\phi$ correlations, we compute
\begin{align}
\langle \phi^\alpha \phi^\beta \rangle &= \sqrt{\det\frac{A+B}{\pi}} \intd\phi \intd\phi^\prime  \delta^{(n)}(\phi - \phi^\prime)\, \phi^\alpha \phi^\beta e^{ -\frac{1}{2} (\phi^\mathrm{T} A \phi + {\phi^\prime}^\mathrm{T} A \phi^\prime + \phi^\mathrm{T} B \phi^\prime + {\phi^\prime}^\mathrm{T} B \phi ) } \nonumber\\
&= \left[ \frac{\partial}{\partial J_\alpha} \frac{\partial}{\partial J_\beta}\, 
\underbrace{\sqrt{\det\frac{A+B}{\pi}} \intd\phi  e^{ -\phi^\mathrm{T} (A+B) \phi + J^\mathrm{T} \phi }}_{= Z[J]} \right]_{J=0} \nonumber\\
&= \left[ \frac{\partial}{\partial J_\alpha} \frac{\partial}{\partial J_\beta} e^{\frac{1}{4} J^\mathrm{T} (A+B)^{-1} J} \right]_{J=0}  = \frac{1}{2} ((A+B)^{-1})^{\alpha\beta}\ .
\end{align}
Thus, we find that $\Gamma^{\alpha\beta}_{\phi\phi} \equiv \langle \{ \phi^\alpha, \phi^\beta \} \rangle = ((A+B)^{-1})^{\alpha\beta}$. Next, we calculate the $\pi\pi$ correlations, using the field representation $\pi^k \to - \i \partial/\partial\phi_k$:
\begin{align}
\langle \pi^\alpha \pi^\beta \rangle &= - \sqrt{\det\frac{A+B}{\pi}} \intd\phi \intd\phi^\prime  \delta^{(n)}(\phi - \phi^\prime)\, \frac{\partial}{\partial\phi_\alpha} \frac{\partial}{\partial\phi_\beta} e^{ -\frac{1}{2} (\phi^\mathrm{T} A \phi + {\phi^\prime}^\mathrm{T} A \phi^\prime + \phi^\mathrm{T} B \phi^\prime  + {\phi^\prime}^\mathrm{T} B \phi )} \nonumber\\
&= \sqrt{\det\frac{A+B}{\pi}} \intd\phi \intd\phi^\prime  \delta^{(n)}(\phi - \phi^\prime)\, \left( A^{\alpha\beta} - ({A^\alpha}_\gamma\phi^\gamma + {B^\alpha}_\gamma{\phi^\prime}^\gamma) ({A^\beta}_\delta \phi^\delta + {B^\beta}_\delta {\phi^\prime}^\delta) \right) e^{ -\frac{1}{2} \phi^\mathrm{T} A \phi + \dots } \nonumber\\
&= \sqrt{\det\frac{A+B}{\pi}} \intd\phi \left( A^{\alpha\beta} - ({A^\alpha}_\gamma + {B^\alpha}_\gamma) ({A^\beta}_\delta + {B^\beta}_\delta) \phi^\gamma\phi^\delta  \right) e^{ -\phi^\mathrm{T} (A+B) \phi} \nonumber\\
&= \sqrt{\det\frac{A+B}{\pi}} \left( \frac{A^{\alpha\beta}}{\sqrt{\det\frac{A+B}{\pi}}} - \frac{1}{2} ({A^\alpha}_\gamma + {B^\alpha}_\gamma) ({A^\beta}_\delta + {B^\beta}_\delta) ((A+B)^{-1})^{\gamma\delta} \right) \nonumber\\
&= A^{\alpha\beta} - \frac{1}{2} \left( A^{\alpha\beta} + B^{\alpha\beta} \right) = \frac{1}{2} \left( A^{\alpha\beta} - B^{\alpha\beta} \right)\ .
\end{align}
Hence $\Gamma^{\alpha\beta}_{\pi\pi} \equiv \langle \{ \pi^\alpha, \pi^\beta \} \rangle = (A-B)^{\alpha\beta}$.
In reverse, we simply find
\begin{align}
A &= \frac{1}{2} \left( \Gamma_{\phi\phi}^{-1} + \Gamma_{\pi\pi} \right)\ , &
B &= \frac{1}{2} \left( \Gamma_{\phi\phi}^{-1} - \Gamma_{\pi\pi} \right)\ .
\end{align}
Alternatively, using the form \eqref{EQ_RHO1}, we can write
\begin{align}
P &= \Gamma_{\pi\pi} , &
Q R^{-1} Q^\mathrm{T} &= \Gamma_{\pi\pi} - \Gamma_{\phi\phi}^{-1} \ . &
\end{align}
Note that due to our assumption of real matrices $A,B$, the $\phi\pi$ correlator can only be imaginary, and must therefore vanish.

\subsection{Complex case}
We now consider the more complicated case where the exponential term in \eqref{EQ_RHO2} is complex, i.e., we assume density matrices of the form
\begin{equation}
\rho(\phi, \phi^\prime) = \sqrt{\det\frac{A+B}{\pi}} e^{-\frac{1}{2} (\phi^\mathrm{T} (A+\i C) \phi + {\phi^\prime}^\mathrm{T} (A-\i C) \phi^\prime + \phi^\mathrm{T} (B+\i D) \phi^\prime + {\phi^\prime}^\mathrm{T} (B-\i D) \phi ) }\ ,
\end{equation}
where the $n\times n$ matrices $A,B,C,D$ are real, with $A,B,C$ being symmetric and $D$ being antisymmetric, following from $\rho^\dagger=\rho$ and the commutativity of the modes $\phi_k$. Again, let us confirm the normalization
\begin{align}
Z = \tr{\rho} &=  \sqrt{\det\frac{A+B}{\pi}} \intd\phi \intd\phi^\prime  \delta^{(n)}(\phi - \phi^\prime)\, e^{-\frac{1}{2} (\phi^\mathrm{T} (A+\i C) \phi + {\phi^\prime}^\mathrm{T} (A-\i C) \phi^\prime + \phi^\mathrm{T} (B+\i D) \phi^\prime  + {\phi^\prime}^\mathrm{T} (B-\i D) \phi) } \nonumber\\
&= \sqrt{\det\frac{A+B}{\pi}} \intd\phi e^{ -\phi^\mathrm{T} (A+B) \phi} 
= 1\ .
\end{align}
The $\phi\phi$ correlator is equivalent to the real case:
\begin{align}
\langle \phi^\alpha \phi^\beta \rangle &= \sqrt{\det\frac{A+B}{\pi}} \intd\phi \intd\phi^\prime  \delta^{(n)}(\phi - \phi^\prime)\, \phi^\alpha \phi^\beta e^{ -\frac{1}{2} (\phi^\mathrm{T} (A+\i C) \phi + {\phi^\prime}^\mathrm{T} (A-\i C) \phi^\prime + \phi^\mathrm{T} (B+\i D) \phi^\prime + {\phi^\prime}^\mathrm{T} (B-\i D) \phi ) } \nonumber\\
&= \left[ \frac{\partial}{\partial J_\alpha} \frac{\partial}{\partial J_\beta}\, 
\underbrace{\sqrt{\det\frac{A+B}{\pi}} \intd\phi  e^{ -\phi^\mathrm{T} (A+B) \phi + J^\mathrm{T} \phi }}_{= Z[J]} \right]_{J=0} \nonumber\\
&= \left[ \frac{\partial}{\partial J_\alpha} \frac{\partial}{\partial J_\beta} e^{\frac{1}{4} J^\mathrm{T} (A+B)^{-1} J} \right]_{J=0}  = \frac{1}{2} ((A+B)^{-1})^{\alpha,\beta}\ .
\end{align}
Again, we find $\Gamma^{\alpha\beta}_{\phi\phi} \equiv\langle \{ \phi^\alpha, \phi^\beta \} \rangle = ((A+B)^{-1})^{\alpha\beta}$. The $\phi\pi$ correlator, on the other hand, is now nonzero:
\begin{align}
\langle \phi^\alpha \pi^\beta \rangle &= -\i \sqrt{\det\frac{A+B}{\pi}} \intd\phi \intd\phi^\prime  \delta^{(n)}(\phi - \phi^\prime)\, \phi^\alpha \frac{\partial}{\partial\phi_\beta} e^{ -\frac{1}{2} (\phi^\mathrm{T} (A+\i C) \phi + {\phi^\prime}^\mathrm{T} (A-\i C) \phi^\prime + \phi^\mathrm{T} (B+\i D) \phi^\prime  + {\phi^\prime}^\mathrm{T} (B-\i D) \phi) } \nonumber\\
&= \sqrt{\det\frac{A+B}{\pi}} \intd\phi \intd\phi^\prime  \delta^{(n)}(\phi - \phi^\prime)\, \phi^\alpha \left( {(\i A-C)^\beta}_\gamma\,\phi^\gamma + {(\i B - D)^\beta}_\gamma {\phi^\prime}^\gamma \right) e^{ -\frac{1}{2} \phi^\mathrm{T} (A+\i C) \phi + \dots } \nonumber\\
&= \sqrt{\det\frac{A+B}{\pi}} {\left( -C - D + \i A + \i B \right)^\beta}_\gamma\, \intd\phi \phi^\alpha \phi^\gamma e^{ -\frac{1}{2} \phi^\mathrm{T} (A+B) \phi } \nonumber\\
&= {\left( -C - D + \i A + \i B \right)^\beta}_\gamma\, \frac{1}{2} ((A+B)^{-1})^{\alpha\gamma} \nonumber\\
&= \frac{\i}{2} \delta^{\alpha\beta} - \frac{1}{2} ((A+B)^{-1}(C-D))^{\alpha\beta}\ .
\end{align}
It follows that $\Gamma^{\alpha\beta}_{\phi\pi} \equiv \langle \{ \phi^\alpha, \pi^\beta \} \rangle = 2 \langle \phi^\alpha\pi^\beta \rangle - \i\delta^{\alpha\beta} = -((A+B)^{-1}(C-D))^{\alpha\beta}$. This also implies $\Gamma_{\pi\phi} = \Gamma_{\phi\pi}^\mathrm{T} = -(C+D) (A+B)^{-1}$, as $D^\mathrm{T}=-D$.
Finally, we compute the $\pi\pi$ correlator:
\begin{align}
\langle \pi^\alpha \pi^\beta \rangle &= - \sqrt{\det\frac{A+B}{\pi}} \intd\phi \intd\phi^\prime  \delta^{(n)}(\phi - \phi^\prime)\,\frac{\partial}{\partial\phi_\alpha}\frac{\partial}{\partial\phi_\beta} e^{ -\frac{1}{2} (\phi^\mathrm{T} (A+\i C) \phi + {\phi^\prime}^\mathrm{T} (A-\i C) \phi^\prime + {\phi^\prime}^\mathrm{T} (B+\i D) \phi + \phi^\mathrm{T} (B-\i D) \phi^\prime ) } \nonumber\\
&= \sqrt{\det\frac{A+B}{\pi}} \intd\phi \left( (A+\i C)^{\alpha\beta} - {(A+B + \i C + \i D)^\alpha}_\gamma\, {(A+B + \i C + \i D)^\beta}_\delta\, \phi^\gamma \phi^\delta \right) e^{ -\frac{1}{2} \phi^\mathrm{T} (A+B) \phi } \nonumber\\
&= (A+\i C)^{\alpha\beta} - {(A+B + \i C + \i D)^\alpha}_\gamma\, {(A+B+ \i C + \i D)^\beta}_\delta\, \frac{1}{2} ((A+B)^{-1})^{\gamma\delta} \nonumber\\
&= (A+\i C)^{\alpha\beta} - \frac{1}{2} (A+B)^{\beta\alpha} - \frac{\i}{2}(C+D)^{\beta\alpha} - \frac{\i}{2} (C+D)^{\alpha\beta} + \frac{1}{2} {(C+D)^\alpha}_\gamma {(C+D)^\beta}_\delta ((A+B)^{-1})^{\gamma\delta} \nonumber\\
&= \frac{1}{2}(A - B)^{\alpha\beta} + \frac{1}{2} \left( (C+D)(A+B)^{-1}(C-D) \right)^{\alpha\beta}\ .
\end{align}
We thus find the full correlator $\Gamma^{\alpha\beta}_{\pi\pi} \equiv \langle \{ \pi^\alpha, \pi^\beta \} \rangle = (A - B)^{\alpha\beta} + ( (C+D)(A+B)^{-1}(C-D))^{\alpha\beta}$. To summarize:
\begin{align}
\Gamma_{\phi\phi} &= (A+B)^{-1}\ , & \Gamma_{\pi\pi} &= A-B + (C+D) (A+B)^{-1} (C+D)^\mathrm{T}\ , \nonumber\\
\Gamma_{\phi\pi} &= -(A+B)^{-1} (C+D)^\mathrm{T}\ , & \Gamma_{\pi\phi} &= - (C+D) (A+B)^{-1}\ .
\end{align}
Again, we can reconstruct the matrices $A,B,C,D$ from the correlators:
\begin{align}
A &= \frac{1}{2} \left( \Gamma_{\phi\phi}^{-1} + \Gamma_{\pi\pi} - \Gamma_{\pi\phi} \Gamma_{\phi\phi}^{-1} \Gamma_{\phi\pi} \right)\ , &
B &= \frac{1}{2} \left(  \Gamma_{\phi\phi}^{-1} - \Gamma_{\pi\pi} + \Gamma_{\pi\phi} \Gamma_{\phi\phi}^{-1} \Gamma_{\phi\pi}  \right)\ ,\nonumber\\
C &= -\frac{1}{2} \left( \Gamma_{\pi\phi} \Gamma_{\phi\phi}^{-1} + \Gamma_{\phi\phi}^{-1} \Gamma_{\phi\pi} \right)\ , &
D &=  -\frac{1}{2} \left( \Gamma_{\pi\phi} \Gamma_{\phi\phi}^{-1} - \Gamma_{\phi\phi}^{-1} \Gamma_{\phi\pi} \right)\ .
\end{align}

\section{Covariance matrices and fermions}
\subsection{Gaussian states}
An even-parity fermionic Gaussian state vector $\ket\psi$ in an $N$-fermion Hilbert space can be parametrized by an antisymmetric $N \times N$ \emph{generating matrix} $A$ in the form
\begin{equation}
\label{EQ_FERM_GAUSS_DEF}
\ket \psi = \frac{1}{\sqrt{Z}} \exp\left( \frac{1}{2} \sum_{j,k=1}^N A_{j,k} \fd_j \fd_k \right) \vacket \ ,
\end{equation}
where $\vacket$ is the fermionic vacuum state vector and $1/\sqrt{Z}$ is a normalization factor. To generalize this to odd-parity states, one would need to ``shift'' the vacuum by a term linear in creation operators $\fd_k$ in the form
\begin{equation}
\vacket \to \intd g B_{j,k}\, \mathrm{g}_j \fd_k \vacket \ ,
\end{equation}
with $\mathrm{g}_k$ being an auxiliary Grassmann variables fulfilling $\{ g_j, \fd_k \} = 0$, $\intd g$ a Grassmann integration over these variables\footnote{A Grassmann integration $\intd{g_k}$ is simply defined by $\intd{g_j} g_k = \delta_{j,k}$, i.e., equivalent to a derivative.}, and $B$ an $M \times N$ matrix. In general, $0 \leq M \leq N$, and we can always ``gauge'' the matrices $A$ and $B$ so that $B A = 0$. 
To define a fermionic covariance matrix for a state of the form \eqref{EQ_FERM_GAUSS_DEF}, we consider the four possible Hermitian operators quadratic in fermionic modes:
\begin{align}
F^1_{j,k} &= \frac{1}{2}\sandwich{\psi}{\fe_j \fe_k - \fd_j \fd_k}{\psi} \ , &
F^2_{j,k} &= \frac{1}{2}\sandwich{\psi}{\fe_j \fd_k - \fd_j \fe_k}{\psi} \ , \\
F^3_{j,k} &= \frac{1}{2}\sandwich{\psi}{\i(\fe_j \fe_k + \fd_j \fd_k)}{\psi}  \ , &
F^4_{j,k} &= \frac{1}{2}\sandwich{\psi}{\i(\fe_j \fd_k - \fe_k \fd_j)}{\psi} \ .
\end{align}
These can be combined into one antisymmetric $2N \times 2N$ covariance matrix $\Gamma$ of \emph{Majorana operators} $\m_k$,
\begin{equation}
\Gamma_{j,k}(\psi) =\bra{\psi}\tfrac{\i\,}{2} [\m_j,\m_k]\ket \psi \ ,
\end{equation}
where the operators are defined via $\fe_k = \m_{2k-1} + \i \m_{2k}$ and  $\fd_k = \m_{2k-1} - \i \m_{2k}$.
Each $2 \times 2$ block of $\Gamma$ can be related to the $F^k$ operators:
\begin{align}
\Gamma_{2j-1,2k-1} &= F^3_{j,k} + F^4_{j,k} - F^3_{k,j} - F^4_{k,j} \ ,&
\Gamma_{2j-1,2k} &= F^1_{j,k} - F^2_{j,k} - F^1_{k,j} + F^2_{k,j} \ ,\\
\Gamma_{2j,2k-1} &= F^1_{j,k} + F^2_{j,k} - F^1_{k,j} - F^2_{k,j} \ ,&
\Gamma_{2j,2k} &= -F^3_{j,k} + F^4_{j,k} + F^3_{k,j} - F^4_{k,j} \ .
\end{align}
Before computing the covariance matrix, we fix the normalization:
\begin{equation}
\braket{\psi}\psi = \frac{1}{Z} \vacbra \exp\left( \frac{1}{2} \sum_{j,k=1}^N A^\dagger_{j,k} \fe_j \fe_k \right) \exp\left( \frac{1}{2} \sum_{j,k=1}^N A_{j,k} \fd_j \fd_k \right) \vacket \overset{!}{=} 1 \ .
\end{equation}
We can resolve this operator product by rewriting it as a Grassmann integration over two set of Grassmann variables $E=(\eta_1,\eta_2,\dots,\eta_N)$ and $\Theta=(\theta_1,\theta_2,\dots,\theta_N)$ that replace the fermionic creation and annihilation operators, fulfilling $\{ \theta_j,\theta_k \} = \{ \eta_j,\eta_k \} =\{ \theta_j,\eta_k \} = 0$. This leads to the equation
\begin{align}
Z &= \intd{E}\intd{\Theta} ( 1 + \Theta^\mathrm{T} E) \exp\left( \frac{1}{2} E^\mathrm{T} A^\dagger E + \frac{1}{2} \Theta^\mathrm{T} A \Theta \right) \nonumber\\
&= \intd{E}\intd{\Theta}
\exp\left[
\frac{1}{2}
\begin{pmatrix}
E \\
\Theta
\end{pmatrix}^\mathrm{T}
\begin{pmatrix}
A^\dagger & \id_N \\
-\id_N    & A
\end{pmatrix}
\begin{pmatrix}
E \\
\Theta
\end{pmatrix}
\right] \nonumber\\
&= \sqrt{\det{
\begin{pmatrix}
-A^\dagger & -\id_N \\
\id_N    & -A
\end{pmatrix}}} \ ,
\end{align}
In order to compute covariance matrices, i.e., two-point functions in fermionic operators, we need to insert two Grassmann-valued source terms $I=(\iota_1,\iota_2,\dots,\iota_{N})$ and $K=(\kappa_1,\kappa_2,\dots,\kappa_{N})$  into the partition function:
\begin{align}
Z[I,K] &= \intd{E}\intd{\Theta}
\exp\left[
\frac{1}{2}
\begin{pmatrix}
E \\
\Theta
\end{pmatrix}^\mathrm{T}
\begin{pmatrix}
A^\dagger & \id_N \\
-\id_N    & A
\end{pmatrix}
\begin{pmatrix}
E \\
\Theta
\end{pmatrix} 
\,+\, 
\begin{pmatrix}
E \\
\Theta
\end{pmatrix}^\mathrm{T}
\begin{pmatrix}
I \\
K 
\end{pmatrix}
\right] \nonumber\\
&= Z \exp\left(\frac{1}{2} \begin{pmatrix}
I \\
K 
\end{pmatrix}^\mathrm{T}
\begin{pmatrix}
A^\dagger & \id_N \\
-\id_N    & A
\end{pmatrix}^{-1}
\begin{pmatrix}
I \\
K 
\end{pmatrix}
\right)  \nonumber\\
&=
Z \exp\left(
\frac{1}{2} 
\begin{pmatrix}
I \\
K 
\end{pmatrix}^\mathrm{T}
\begin{pmatrix}
A (A^\dagger A + \id_N)^{-1} & -(A A^\dagger + \id_N)^{-1} \\
(A A^\dagger + \id_N)^{-1}  & A^\dagger (A A^\dagger + \id_N)^{-1}
\end{pmatrix}^{-1}
\begin{pmatrix}
I \\
K 
\end{pmatrix}
\right) \ .
\end{align}
We can now explicitly compute fermionic observables, such as
\begin{align}
F^1_{j,k} &= \frac{1}{2}\sandwich{\psi}{\fe_j \fe_k - \fd_j \fd_k}{\psi} = \frac{1}{Z} \left[ \frac{\partial^2 Z[I,K]}{\partial\iota_j\partial\iota_k} - \frac{\partial^2 Z[I,K]}{\partial\kappa_j\partial\kappa_k} \right]_{I=K=0} \nonumber\\
&= \frac{1}{2}\left( A^\dagger (A A^\dagger + \id_N)^{-1} - A (A^\dagger A + \id_N)^{-1} \right)_{j,k}\ .
\end{align}
Note the change in sign due to anticommutativity of the partial derivatives.
Using this approach, we determine all observables $F^k$ to be 
\begin{align}
F^1 &= \Re\left( A (A^\dagger A + \id_N)^{-1} \right) &
F^2 &= \Re((A^\dagger A + \id_N)^{-1}) - \id_N \\
F^3 &= \Im\left( A (A^\dagger A + \id_N)^{-1} \right) &
F^4 &= -\Im((A^\dagger A + \id_N)^{-1})
\end{align}
The Majorana covariance matrix $\Gamma$ can then be constructed from the $F^k$.

\end{document}